\newcommand{\tauconst}{5000}
\newcommand{\query}{[INSERIRE LA QUERY CON NOTAZIONE CTL]}
\newcommand{\mT}{\ensuremath{\mu_T}}
\newcommand{\vT}{\ensuremath{\sigma_T}}
\newcommand{\mH}{\ensuremath{\mu_H}}
\newcommand{\vH}{\ensuremath{\sigma_H}}
\newcommand{\K}{\ensuremath{K}}
\newcommand{\expdel}{\ensuremath{\lambda}}

\section{Properties}
In this section we are going to describe the properties that we have verified over the system to ensure its correctness and to analyse its behaviour. Unless otherwise specified, we ran our verifications with a time bounded to \tauconst \space time units, in order to emulate a realistic environment when fully operational.

During the verification phase we have used the default statistical parameters of \UPPAAL. It was in reasonable considering a confidence of 95\% for verification purposes. We also considered the other statistical parameters as reasonable [NON MI PIACE QUESTA FRASE FORSE LA TOLGO]. Those parameters with the respective values are reported in the Table \ref{tab:statparam}.

\begin{table}[h]
    \centering
        \begin{tabular}{|c c|} 
            \hline
            Parameter & Value \\ [0.5ex] 
            \hline\hline
            $\pm\delta$ & 0.01 \\
            $\beta$ & 0.05 \\
            $\alpha$ & 0.05 \\
            $\epsilon$ & 0.05 \\
            $\mu_0$ & 0.9 \\
            $\mu_1$ & 1.1 \\
            Trace resolution & 4096 \\
            Discretization step & 0.01 \\ [0.5ex] 
            \hline
        \end{tabular}
        \caption{Statistical parameters of \UPPAAL}
        \label{tab:statparam}
\end{table}

\subsection{Full queue [UN NOME FIGO PER IL TITOLO?]}
This is the most important property to be verified, in order to understand if the system reacts in a proper way to several configurations. The system should sustain a high number of tasks generated during the execution.

In this section we will stress the system by changing the delays and the grid configuration. We will also try adding some robots into the system, so that we will be able to see the main differences, comparing them and understanding what are the trade-off to take into account. In particular, the parameters which I will refer to are listed here:
\begin{itemize}
    \item \mT: the average delay of the task generation
    \item \vT: the standard deviation of the task generation
    \item \mH: the average delay when a human picks an item
    \item \vH: the standard deviation when a human picks an item
    \item \K: the \emph{deterministic} value of the time that passes between each move of the robot
    \item \expdel: the delay with which the robot claims a task. Note that this parameter has not been changed in any configuration, as we considered $\expdel= 1$ a realistic situation [MIGLIORARE FRASE].
\end{itemize}

We analysed three different scenarios. In the first one, we analysed the behaviour og the system with a \texttt{10x10} grid and 3 robots. In the second one, we increased the number of robots to 6, maintaining the same grid dimension. Finally, we observed how the system reacts to a \texttt{8x12} grid, being it rectangular and not squared. In this case, we used 4 robots, to simulate an average situation [?].

For all the scenarios we used the same query, that can be found in the tab \emph{Verifier} of \UPPAAL. It is equivalent to the CTL formula:

\begin{equation}
    SOSTITUIRE ->  \lnot \square \vee \vdash \lozenge \wedge \models  <- SOSTITUIRE
\end{equation}
Where $TAU$ is the upper bound limit. As I mentioned before, $TAU = \tauconst$ in each verification.

In order to offer a general overview of all the vericiations, we collected all the results in the Table \ref{tab:scenonetable}. We will use this table during the description of each scenario, describing why we used certain values and the purpose of each verification.

\begin{table}[h]
    \centering
        \begin{tabular}{| c || c c c c c |} 
            \hline
             & Scen 1.a & Scen 1.b & Scen 1.c & Scen 1.d & Scen 1.e \\ [0.5ex] 
            \hline\hline
            \mT & 40 & 150 & 40 & 30 & 30 \\
            \vT & 5 & 10 & 5 & 5 & 5 \\
            \mH & 3 & 6 & 6 & 3 & 6 \\
            \vH & 2 & 4 & 4 & 2 & 4 \\
            \K & 1 & 1 & 2 & 1 & 1 \\
            \expdel & 1 & 1 & 1 & 1 & 1 \\
            \hline\hline
            $p\in [\%]$ &  $[24.8,34.8]$ &  $[41.4,14.1]$ &  $[90.2,100]$ & $[76.4,86.4]$ & $[90.2, 100]$ \\ [0.5ex] 
            \hline
        \end{tabular}
        \caption{General tabel [NAME]}
        \label{tab:scenonetable}
    \end{table}

\subsubsection{First Scenario}
The first scenario is the simplest one. In this case, there were 3 robots and the dimension of the grid was \texttt{10x10}.

Given that three robots are not able to handle a large number of tasks, we firstly chose to give a high average time for the task generation delay (\mT) while maintaining a general realistic behaviour between the robots and the human (\mH). We assumed that the human is slower than the robots and that it has a significant high deviation standard (\vH). We also assumed the task generation to be quite constant, keeping a small deviation standard (\vT), in order to avoid creating a too floating situation. All these parameters are collected in the scenario \emph{Scenario 1.a} of the Table \ref{tab:scenonetable}.

The aforementioned assumptions could bring to the conclusion that, probably, the robots will be able to handle the tasks generated quite easily. However, the verification results show that there is about the 30\% of probability that the queue will reach the maximum size.
\\

As a consequence, we decreased the task generation frequency until the result reached a low probability. We did that to understand what is, in this configuration, the maximum frequency that the system can handle.

In these scenario we focused on the average delay time \mT \space and we changed the other parameters only in order to give a realistic behaviour to the overall system. [DEVO CAPIRE COME SPIEGARE BENE IL MOTIVO PER CUI ABBIAMO SOLO CAMBIATO \mT]

We used the values indicated at Table \ref{tab:scenonetable} under \emph{Scenario 1.b} and we considered the resulting interval sufficient to indicate $\mT = 150$ as the maximum frequency [NON MI PIACE LA FRASE, praticamente ho spiegato che grazie a quel range di probabilità, a noi va bene così].
\\

Finally, we tried stressing the system in order to understand when it can no more handle the situation. We did it in three ways:
\begin{itemize}
    \item Firstly, we tried increasing \K, with the aim of delaying the robots. It was enough to double the delay in order to reach the maximum probabilistic range as shown at Table \ref{tab:scenonetable} under \emph{Scen 1.c}.
    \item Subsequently we tried decreasing the task generation delay. At Table \ref{tab:scenonetable} under \emph{Scen 1.d}, we can see that, just decreasing the generation delay by $10$, the probabilistic range increased of about the $50\%$ with respect to the \emph{Scen 1.a}.
    \item Finally, we modified the previous analysis. We kept the same values, except for delaying a little the human (Table \ref{tab:scenonetable}, \emph{Scen 1.e}). It is enough to delay the human some time units in order to reach the maximum possible range.
\end{itemize}

In conclusion, we would like to underline the main characteristics identified in the first scenario. We can notice that, in order to stress the system, decreasing a few time units the task generation delay is enough to have the system unable to handle the configuration. The same can be said for increasing the human delay.

On the other hand, the system begins to behave in a smooth way only when we set quite high the task generation delay. We will better understand in section \ref{secondscenario} that this behaviour is typical when there are few robots in the system.

\subsubsection{Second Scenario} \label{secondscenario}
In this scenario, we will analyse a configurations with six robots and a \texttt{10x10} grid.

The intuitive reasoning could be that the more robots there are in the system, the smoothest it will be in handling a non trivial number of tasks. Moreover, it should be improbable for the robots to find `traffic' in the map, as they are much less than the total number of cells available.

However, the results of the verification are far worser than the expected. In order to compare the two scenarios, we used the same values of the \emph{Scen 1.a} at Table \ref{tab:scenonetable}. The resulting probability the one showed at  The probability of filling up the queue is greater than before: it lies in the range \texttt{[0.601786,0.701727]}.

We therefore did the same steps as in the previous scenarios, to see if there were some differences in the distribution of the probabilities. After using the same values of Tabel \ref{tab:lowbound}, the resulting range of probability was \texttt{[0.199284,0.299149]}, about the 15\% more than the result with three robots. Considering that the verification results showed in the previous paragraph were about the 35\% higher than the case with three robots, we can conclude that with more robots it is faster improving the situation. It also makes sense, because the less tasks there are, the less `traffic' there is.

In order to reach about the same interval of probabilities, we also ran a verification with $mT = 200$ and, producing a resulting interval of \\ \texttt{[0.0619394,0.16173]}.

\subsubsection{Third Scenario}
The last scenario that we would like to present is less specific. We stressed an environment with a \texttt{9x9} grid and 4 robots, in order to combine together some properties.

Here we concentrated more on the reactions of the system when changing the standard deviations and the human average delay time.

We will only report the relevant results.

In questo pezzo ho fatto una prova normale, poi ho aumentato la deviazione standard prima del task e poi del human. In entrambi i casi, va a incidere solo se superi il valore della media, e va incidere poco. Comunque, per l'umano aumenta la probabiiltà di errore e invece per il task diminusice, giustamente.

Mi sveglio tra 3 ore e mezza e finiamo. Ho vomitato tutto ciò che c'è da scrivere, almeno da quel punto di vista ho finito, poi tocca abbellirlo.

Inoltre, non sono riuscito a fare altre proprietà non tanto per mancanza di tempo, quanto perché non sono riuscito proprio neanche a pensare a cose sensate (e la cosa del pods\_unavailable non mi funzionava)