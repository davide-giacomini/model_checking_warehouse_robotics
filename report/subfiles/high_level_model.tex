\begin{abstract}\addcontentsline{toc}{section}{Abstract}
The development of cutting-edge technologies in automated warehouses has taken outstanding steps in the last few years. 
This fast-paced environment requires a high level of reliability, in terms of speed and interaction between the many system components.
Formal verification plays an essential role in ensuring an error-free environment, by analysing synthetic models, that capture the main features of the real-world problem.\\
The Model Checking of Warehouse Robotics project employs \UPPAAL\ \cite{uppaal} to model a simplified automated warehouse environment. The goal is to verify some significant properties in different scenarios, in order to show the correctness of the model.
\end{abstract}

\section{Introduction}
The Model Checking of Warehouse Robotics project is divided in two main focus areas: first, the efforts are placed in modelling the environment; then the target becomes the verification of significant properties, applied to the aforementioned model.

The goal is the handling of multiple tasks: each task contains the information about a specific item (placed in a pod inside the warehouse) that needs to be delivered. 
In particular, the high level model consists in a set of three different components, specifically the human, the robots and the task queue generator.
The environment where these components interact is a rectangular grid (i.e., the warehouse floor plan). It is possible to identify on the map an entry point, a delivery point and a set of pods.

\begin{tabularx}{\textwidth}{lX}
\textbf{Entry point} & It is the initial slot where all the robots are placed at the beginning of the simulation.\vspace{0,2cm}\\
\textbf{Delivery point} & It is the slot where the items need to be delivered.\vspace{0,2cm}\\
\textbf{Pod} & It includes some items and it can be moved by the robots.\vspace{0,2cm}\\
\end{tabularx}

\noindent Specifically, these are the main features of the components:
\vspace{0,2cm}
\begin{tabularx}{\textwidth}{lX}
\textbf{Human} & The job of the operator is to pick up an item once a robot comes into the human working station. The operator is kept by default at the edge of the system.\vspace{0,2cm}\\
\textbf{Robot} & It is in charge of the delivery of a specific pod to a human, according to what specified in a task.\vspace{0,2cm}\\
\textbf{Task queue} & It periodically generates new tasks and places them into a First-In-First-Out (FIFO) queue.\vspace{0,2cm}\\
\end{tabularx}

Non-deterministic delays in the actions performed by the components have been included in order to create a model closer to a real-world situation, using \UPPAAL \ SMC.